\documentclass[12pt]{article}
\usepackage[english]{babel}
\usepackage{cmbright}
\usepackage{enumitem}
\usepackage{fancyhdr}
\usepackage{fontawesome5}
\usepackage{geometry}
\usepackage{hyperref}
\usepackage[sf]{libertine}
\usepackage{microtype}
\usepackage{paracol}
\usepackage{supertabular}
\usepackage{titlesec}
\usepackage{amsmath}
\usepackage{amssymb} 
\usepackage{amsthm}  

\hypersetup{colorlinks, urlcolor=black, linkcolor=black}

% Geometry
\geometry{hmargin=1.75cm, vmargin=2.5cm}
\columnratio{0.65, 0.35}
\setlength\columnsep{0.05\textwidth}
\setlength\parindent{0pt}
\setlength{\smallskipamount}{8pt plus 3pt minus 3pt}
\setlength{\medskipamount}{16pt plus 6pt minus 6pt}
\setlength{\bigskipamount}{24pt plus 8pt minus 8pt}

% Style
\pagestyle{empty}
\titleformat{\section}{\scshape\LARGE\raggedright}{}{0em}{}[\titlerule]
\titlespacing{\section}{0pt}{\bigskipamount}{\smallskipamount}
\newcommand{\heading}[2]{\centering{\sffamily\Huge #1}\\\smallskip{\large{#2}}}
\newcommand{\entry}[4]{{{\textbf{#1}}} \hfill #3 \\ #2 \hfill #4}
\newcommand{\tableentry}[3]{\textsc{#1} & #2\expandafter\ifstrequal\expandafter{#3}{}{\\}{\\[6pt]}}

\begin{document}

\vspace*{\fill}

\begin{paracol}{2}

	% Name & headline
	\heading{Yago Iglesias Vázquez}

	\switchcolumn

	% Identity card
	\vspace{0.01\textheight}
	\begin{supertabular}{ll}
		\footnotesize\faPhone & +33 7 82 29 56 65 \\
		\footnotesize\faEnvelope & \href{mailto:yago.iglesias.vazquez@gmail.com}{yago.iglesias.vazquez@gmail.com} \\
		\footnotesize\faGithub & \href{https://github.com/Yag000}{github.com/Yag000} \\
	\end{supertabular}

	\bigskip
	\switchcolumn*

	\section{Formation}

	\entry{Université Paris Cité}{Double licence Mathématiques Informatique}{Paris, France}{2021 -- présent}
	\medskip

	\switchcolumn

	\section{Habilidades}
	\begin{supertabular}{rl}
		\tableentry{\footnotesize\faCode}{Rust \textperiodcentered{} Ocaml \textperiodcentered{} C }{}
		\tableentry{}{Python \textperiodcentered{} Java \textperiodcentered{} Bash}{}
		\tableentry{}{}{}

		\tableentry{\footnotesize\faLanguage}{Espagnol \textperiodcentered{} Langue maternelle}{}
		\tableentry{}{Français \textperiodcentered{} Avancée}{}
		\tableentry{}{Anglais \textperiodcentered{} Avancée}{}
	\end{supertabular}

	\switchcolumn*

	\section{Expérience}

	\entry{CiTIUS}{Stage de recherche: Caractérisation d'objets avec LiDAR}{ }{julliet 2023}
	\begin{itemize}[noitemsep,leftmargin=3.5mm,rightmargin=0mm,topsep=6pt]
		\item L'objectif était de développer des algorithmes pour mesurer avec précision des objets avec un seul balayage LiDAR grâce à la réflexion de l'objet sur un miroir bien placé.
	\end{itemize}



	\medskip
	\switchcolumn

	\section{Autres intêtets}


	\textbf{Journées Franciliennes de Programmation}\\
	\noindent Participation en 2022 et 2023

	\textbf{SWERC}\\
	\noindent Participation en 2024
	\bigskip

	%TODO: Add lambda

\end{paracol}

\section{Projets de programmation}

\begin{paracol}{2}
	\entry{\href{https://github.com/Yag000/chimpanzee}{Chimpanzee}}{Interpréteur, compilateur et formateur de Monkey en Rust}{ }{ }
	\medskip
	\switchcolumn

	\entry{\href{https://github.com/Yag000/jsh-tyy}{jst-tyy}}{Job control shell en C développé pour le cours de Systèmes d'exploitation}{ }{ }
	\medskip
	\switchcolumn

	\entry{\href{https://moule.informatique.univ-paris-diderot.fr/iglesias/U3bbU3c0}{$\pmb\lambda\pi$}}{Implémentation des types dépendants en OCaml}{ }{ }
	\medskip
	\switchcolumn






\end{paracol}



\vspace*{\fill}

\end{document}
